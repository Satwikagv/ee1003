\documentclass{beamer}
\usetheme{Madrid}

\usepackage{amsmath,amssymb,amsfonts,amsthm}
\usepackage{graphicx}
\usepackage{listings}
\usepackage{gensymb}
\usepackage[utf8]{inputenc}
\usepackage{hyperref}
\usepackage{gvv}

\begin{document}

\title{9.3.12B}
\author{EE24BTECH11027 - G.V.Satwika}
\date{}
\frame{\titlepage}

\begin{frame}
\frametitle{Problem statement}
Plot the solution of the differential equation: 
\begin{align}
    y^{\prime\prime} +xy^\prime + xy = x. 
\end{align}
\end{frame}
\begin{frame}[allowframebreaks]
\frametitle{Solution}
To plot the curve of the given differential equation $\brak{1}$ we can do it using the method of finite differences which is a numerical technique for solving complex differential equations by approximating derivatives with differences.\\
The approximated forward derivative of $y\brak{x}$ is given as:\\
\begin{align}
    y^\prime_n\approx\frac{y_{n+1}-y_n}{h}
\end{align}
On rearranging we get,
\begin{align}
    y_{n+1}=y_n+y^\prime_n\brak{h}
\end{align}
And also 
\begin{align}
    x_{n+1}=x_n+h
\end{align}
The approximated forward derivative of second order of $y\brak{x}$ is given as:\\
\begin{align}
    y^{\prime\prime}_n\approx \frac{y^\prime_{n+1}-y^\prime_n}{h}
\end{align}
Substitute eq $\brak{2}$ in eq $\brak{5}$ we get,
\begin{align}
    y^{\prime\prime}_n\approx\frac{y_{n+2}-2y_{n+1}-y_n}{h^2}
\end{align}
Substitute  eq $\brak{2}$ and eq $\brak{6}$ in eq $\brak{1}$ and on reaaranging we get,
\begin{align}
    y_{n+2}=y_{n+1}\brak{2-hx_n} +y_n\brak{1+hx_n-h^2x_n}+h^2x_n
\end{align}
We need to assume two initial conditions as it is a second order differential equation. \\So here we assume the initial conditions as 
\begin{align}
    x_0=0\\y_0=0\\y^\prime_0=1\\h=0.1
\end{align}
substitute eq $\brak{8}$, eq $\brak{9}$ and eq $\brak{10}$ in eq $\brak{1}$\\ we get 
\begin{align}
    y^{\prime\prime}\brak{0}=0
\end{align}
Substitute eq $\brak{10}$ in eq $\brak{3}$
\begin{align}
    y_1=y_0+y^\prime_0(0.1)\\
    y_1=0.1
\end{align}
For the rest of the points use eq $\brak{7}$ we get the other points.
\end{frame}
\begin{frame}
\frametitle{Plot}
\begin{figure}[htbp]
	\includegraphics[width=0.75\columnwidth]{figs/fig.png}
\end{figure}
\end{frame}
% Define colors for syntax highlighting
\definecolor{codegreen}{rgb}{0,0.6,0}
\definecolor{codegray}{rgb}{0.5,0.5,0.5}
\definecolor{codepurple}{rgb}{0.58,0,0.82}
\definecolor{backcolour}{rgb}{0.95,0.95,0.92}
% Settings for the C code
\lstset{
    language=C,
    basicstyle=\footnotesize\ttfamily,
    backgroundcolor=\color{backcolour},
    commentstyle=\color{codegreen},
    keywordstyle=\color{blue},
    numberstyle=\tiny\color{codegray},
    stringstyle=\color{codepurple},
    breakatwhitespace=false,
    breaklines=true,
    captionpos=b,
    keepspaces=true,
    numbers=left,
    numbersep=5pt,
    showspaces=false,
    showstringspaces=false,
    showtabs=false,
    tabsize=2
}
\begin{frame}[fragile,allowframebreaks]
\frametitle{C Code}
\lstinputlisting[label=mycode1]{codes/b.c}
\end{frame}
% Define colors for syntax highlighting
\definecolor{codegreen}{rgb}{0,0.6,0}
\definecolor{codegray}{rgb}{0.5,0.5,0.5}
\definecolor{codepurple}{rgb}{0.58,0,0.82}
\definecolor{backcolour}{rgb}{0.95,0.95,0.92}

% Python style for highlighting
\lstset{
    language=Python,
    basicstyle=\ttfamily\small,
    keywordstyle=\color{blue},
    stringstyle=\color{codepurple},
    commentstyle=\color{codegreen},
    backgroundcolor=\color{backcolour},
    breaklines=true,
    breakatwhitespace=true,
    tabsize=4
}

\begin{frame}[fragile,allowframebreaks]
\frametitle{Python Code}
\lstinputlisting[label=mycode1]{codes/b.py}
\end{frame}
\end{document}
