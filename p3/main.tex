\documentclass{beamer}
\usetheme{Madrid}

\usepackage{amsmath,amssymb,amsfonts,amsthm}
\usepackage{graphicx}
\usepackage{listings}
\usepackage{gensymb}
\usepackage[utf8]{inputenc}
\usepackage{hyperref}
\usepackage{gvv}

\begin{document}

\title{10.3.6.1.7}
\author{EE24BTECH11027 - G.V.Satwika}
\date{}
\frame{\titlepage}

\begin{frame}[fragile,allowframebreaks]
\frametitle{Problem statement}
Find the PMF of the event - a multiple of 3, when a die is thrown.\\ 
\textbf{Solution:}\\
\textbf{Theoretical solution:}\\
	\begin{itemize}
		\item We define a random variable $X$ as $X$ : outcome of the die when thrown, is a multiple of 3.
		\item The possible outcomes of a die when thrown are \cbrak{1,2,3,4,5,6}. From these, the multiples of 3 are 3 and 6.
		\item Therefore $X$ can take the values of 3 and 6. So, the probability space for $X$ is \cbrak{3,6} and the outcomes are equally likely.\\
	\end{itemize}
%Let say the probability that $X=3$ is $p$ and $X=6$ is $1-p$.\\
So, the $PMF$ of $X$ is 
\begin{align}
    P\brak{X=x}= \begin{cases}
        \frac{1}{2}, & x=3,6\\
        0, & \text{otherwise}
    \end{cases}
\end{align}
\textbf{Simulation:}
	\begin{itemize}
		\item We can simulate this event by rolling a die multiple times and keeping only the outcomes which are multiples of 3.
		\item We count the number of 3 and 6 and normalize them.
		\item The more number of rolls gives the more accuracy in simulation.
	\end{itemize}
The below graph shows the comparison between both the simulation and theoretical probabilities.\\
\textbf{Plot:}\\
\begin{figure}[h!]
   \centering
   \includegraphics[width=1\columnwidth]{figs/fig.png}
   \caption{PMF of the event a multiple of 3 when a die is thrown}
\end{figure}
\end{frame}
\end{document}
